    %----------------------------------------------------------------------------------------
%	ABSTRACT PAGE
%----------------------------------------------------------------------------------------
\ifnum \value{print}>0 {\setcounter{page}{7}} \fi

\addtotoc{Abstract} % Add the "Abstract" page entry to the Contents 
\addtocontents{toc}{} % Add a gap in the Contents, for aesthetics
\thispagestyle{plain}
\abstract{
\setstretch{1.5}
    %%%%%%%%%%%%%%%%%
\vspace*{0.3in}

Control systems model the complex interactions between the constantly changing quantities and offer a mathematical framework for explaining the world around us. The recent increase in the amount of data and computational power have given rise to Data-Driven techniques, significantly reshaping the theory of control systems by taking over the classical geometrical approaches. This study investigates the newly introduced informativity approach to control problems, tracing its roots in behavioral theory and exploring its applications in analysing various control properties. We delve into the property of dissipativity, an important link between physics and control systems, and extend the results of informativity for dissipativity using the Matrix S-Lemma to systems with only input-output data. Lastly, the problem of robustness of such approaches in the presence of adversarial attacks is discussed. 

The dissertation consists of a total of six chapters -- the first chapter provides an \emph{Introduction} to the research topic, the next chapter is a comprehensive \emph{Literature Review} followed by the necessary pre-requisites of \emph{Behavioral Theory} which views the dynamical systems as a family of trajectories and \emph{Discrete Systems} theory which is very relevant in the development of Data-Driven Analysis and Control.
Then, we come to the Informativity approach, covering its fundamentals and applications in Dissipativity analysis. Further, we discuss the limitations of this approach by assessing its robustness against Adversarial Attacks. Finally, the Conclusion chapter wraps up the topic and discusses future work in the area.

% This report is divided into parts: A comprehensive Literature Review followed by the necessary pre-requisities of Behavioral Theory where a Dynamical System is simply viewed as a family of trajectories. Then, we define various important properties of the system in the Behavioral framework. This approach is very relevant in the development of Discrete System theory which we will incorporate. Then we come to the Informativity approach to Data-Driven analysis, covering its fundamentals and applications in Dissipatity analysis. Finally, we discuss the limitations of this approach by evaluating its robustness against Adversarial Attacks

\vspace*{2cm}

\textit{Index Terms} --- Behavioral Theory, Data-Driven analysis, Data-Driven control, Informativity approach, Adversarial Attacks.
}
\thispagestyle{plain}
