\section{Transition to the Discrete Framework}%\doublespacing % Main chapter title
\subsection*{Notation}
Let's introduce some important notation for the next topics
\begin{itemize}
\item $\mathbb{S}^n$ is the set of all symmetric matrices in $\mathbb{R}^{n\times n}$
% \item $\op{col}(x,y)$ denotes the column vector $\bbm x\\y\ebm$
\item $M\gi$ is the Moore-Penrose pseudo-inverse of a real matrix $M$
\item $M^{\sharp}$ is the right-inverse of a full row rank matrix $M$
\item For a general signal $\bv$ and $k\in \mathbb{Z}$, $t, T\in \mathbb{N}$,\\
$V_{k, k+T}$ is its restriction to the interval $\{k,k+1,\ldots,k+T\}$ as $V_{k, k+T}=\bbm v(k)\\\vdots \\v( k+T)\ebm$
\item $V_{k,t,N}$ is the Hankel matrix associated to $\bv$ given by
\[V_{k,t,N}
=
\begin{bmatrix}
v(k) & v(k+1) & \cdots & v(k+N-1)\\
v(k+1) & v(k+2) & \cdots & v(k+N)\\
\vdots & \vdots & \ddots & \vdots\\
v(k+t-1) & v(k+t)  & \cdots & v(k+t+N-2)
\end{bmatrix}\]
\end{itemize}
\subsection{Inferring the System from the Data}
We have seen analysis of continuous systems using the geometrical and behavioural approach, let us now study discrete systems which will help us to analyse data-driven techniques. A discrete-time linear system is given by
\begin{subequations}\label{ch5:e:lin-sys}
\begin{align}
\bmx(t+1)&=A \bmx(t)+B \bmu(t)\label{lin.sys.1}\\
\bmy(t)&=C \bmx(t)+D \bmu(t)\label{lin.sys.2}
\end{align}
\end{subequations}
where $\bmx(t)\in\mathbb{R}^n, \bmu(t)\in\mathbb{R}^m, \bmy(t)\in\mathbb{R}^p$ and $A,B,C,D$ are real matrices. We assume that the system is controllable and observable.

Over the time $[0,t-1]$, the input-output response of the system can be expressed as
\be
\label{fund.eqn}
\bbm
U_{0,t-1}\\
\hline
Y_{0,t-1}
\ebm
=
\begin{bmatrix}
I_t & 0_{tm\times n}\\
\hline
\mathcal{T}_t & \mathcal{O}_t
\end{bmatrix}
\bbm
U_{0,t-1}\\
\hline
X_{0,0}
\ebm
\ee
where $\mathcal{T}_t$ is the order $t$ Toeplitz matrix and $\mathcal{O}_t$ is the order $t$ observability matrix,
\begin{eqnarray*} \label{toep.obs.matrx}
\mathcal{T}_t := 
\begin{bmatrix}
D & 0 & 0 & \cdots & 0 \\
CB & D & 0 &\cdots & 0\\
CAB & CB & D & \cdots & 0\\
\vdots &\vdots &\vdots &\ddots & \vdots \\
CA^{t-2}B & C A^{t-3} B & C A^{t-4} B & \cdots & D\\
\end{bmatrix}\qquad
\mathcal{O}_t := 
\begin{bmatrix}
C\\
CA\\
\vdots\\
C A^{t-1}
\end{bmatrix}
\end{eqnarray*}
Now, the same response can be written using Hankel matrices by considering differents states of $\bx$ as initial states and applying \eqref{fund.eqn}
\be\label{eq.Htnd}
\begin{bmatrix}
U_{0,t,T-t+1}\\
\hline
Y_{0,t,T-t+1}
\end{bmatrix}=
\begin{bmatrix}
I_{tm} & 0_{tm\times n}\\
\hline
\mathcal{T}_t & \mathcal{O}_t
\end{bmatrix}
%
\begin{bmatrix}
U_{0, t,T-t+1} \\
\hline
X_{0, 1, T-t+1}
\end{bmatrix}
\ee
Now, when the actual values are unavailable, we replace those values of vectors by data to give us the following
\be\label{eq.Ht}
\begin{bmatrix}
(U_d)_{0,t,T-t+1}\\
\hline
(Y_d)_{0,t,T-t+1}
\end{bmatrix}=
\begin{bmatrix}
I_{tm} & 0_{tm\times n}\\
\hline
\mathcal{T}_t & \mathcal{O}_t
\end{bmatrix}
%
\begin{bmatrix}
(U_d)_{0, t,T-t+1} \\
\hline
(X_d)_{0, 1, T-t+1}
\end{bmatrix}
\ee
This condition on data is the basis of control law design using data as we will see in the next section.
\subsection{Persistently Exciting Data}
\begin{definition}{\cite{willems2005note}}
A signal $\bv_{[0, T-1]}\in\mathbb{R}^n$ is persistently exciting of order $L$ if the matrix $V_{0,L,T-L+1}$ has full rank $nL$.
\end{definition}
\begin{lemma}\label{lem:willems}
{\rm \cite{willems2005note}, Willem's Fundamental Lemma}
For the system \eqref{lin.sys.1}, if the input $u_{d, [0, T-1]}$ is persistently exciting of order $n+t$, then the following condition holds
\be\label{ch2:eq:inf for sys ident}
{\rm rank}
\begin{bmatrix}
U_{0, t,T-t+1} \\
\hline
X_{0, 1, T-t+1}
\end{bmatrix}
=
n+ tm
\ee
\end{lemma} 
Using this lemma, a system can be uniquely determined from data when the input is persistently exciting.
We also say, that such data is \emph{informative} for system identification. The topic for informativity will be our next discussion.