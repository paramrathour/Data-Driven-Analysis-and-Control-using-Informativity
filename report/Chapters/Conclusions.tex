\chapter{Conclusion}\doublespacing % Main chapter title
\label{chap:attacks} % Change X to a consecutive number; for referencing this chapter elsewhere, use \ref{ChapterX}
An introduction to the informativity approach was discussed and a new result for dissipativity using the Matrix S-lemma was developed. Though, there are certain limitations and possibilities of future research.

An obvious limitation is that the dissipativity approach works with only controllable and observable systems due to their need in the Willem's fundamental lemma. There are few works extending the result to such domains such as \cite{9551767}.

This approach deals with identifying the necessary and sufficient conditions for informativity. Such conditions are strong, in the sense that they assume the data and informativity can be interchanged. Instead, another situation can be analysed where data contains more information, giving rise to only sufficient conditions. These conditions can computationally perform better albeit with the loss of strong theoretical guarantees. 

The informativity approach has three parts: the model class, the control objective, and noise model. While we have discussed on varying the control objective, the possibilities of analysis using different model class is endless due to the richer nature of nonlinear systems such as bilinear systems \cite{Bisoffi2020,Markovsky2022}, polynomial systems \cite{Guo2020,Guo2022} or rational systems \cite{Strasser2021}. Different noise models can also be considered by incorporating measurement noise \cite{Persis2020} or sample-bounded noise \cite{vanWaarde2022b}.
\section{Attacks}
This study primarily explored new avenues in the vulnerability of direct data-driven control. Further research can be done on different types of data, systems, attacks or adversaries. Investigating defence mechanisms such as detection of adversarial perturbations \cite{metzen2017on} is also crucial.
%Future research directions include theoretical analysis of DGSM.
%Additionally, novel defense techniques other than regularization are needed for reliable direct data-driven control.
Finally, a more comprehensive understanding of the approach and its vulnerabilities are needed for reliable data-driven control.